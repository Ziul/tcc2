\section{Algorítimos de  \textit{string matching}} % (fold)
\label{sec:algoritimos_de_textit}

\subsection{Algorítimos de \textit{string matching} exatos} % (fold)
\label{sub:algoritimos_de_textit}

Há inúmeros exemplos possíveis de algorítimos voltados para \textit{string matching} exatos, mas iremos citar e apresentar dois dos mais usuais.

\subsubsection*{Rabin-Karp} % (fold)
\label{ssub:rabin_karp}

Criado  em 1987  por  Richard M. Karp e Michael O. Rabin, o algoritmo faz de uso de uma \textit{hash} no intuito de agilizar a busca por um padrão em um texto. Para condições de busca em um texto de tamanho $n$ por um padrão de tamanho $p$, o algoritmo de Rabin-Karp pode ter em melhor caso ordem de complexidade $\Theta(n-m+1)$\cite{paulo2015algoritmos}\footnote{Onde $m=n+p$}. Em pior caso, a ordem de complexidade chega a $\Theta((n-m+1)m)$, mesma ordem de complexidade que uma busca comum/força bruta\cite{paulo2015algoritmos}. Todavia, a situação de pior caso só ocorre quando a \textit{hash} não é propriamente calculada. No código \ref{rkalgoritm} apresentamos de forma simples como implementar o algoritmo de Rabin–Karp.

\begin{lstlisting}[language=Bash,label=rkalgoritm,caption={Algoritmo de Rabin–Karp}]
function RabinKarp(string s[1..n], string pattern[1..m])
	hpattern := hash(pattern[1..m]);  hs := hash(s[1..m])
	for i from 1 to n-m+1
		if hs = hpattern then
			if s[i..i+m-1] = pattern[1..m] then
				return i
		hs := hash(s[i+1..i+m])
	return not found
\end{lstlisting}

Observe que na linha $7$ do código \ref{rkalgoritm} temos o calculo de uma nova \textit{hash}, porém a variável {\code hs} já possui a \textit{hash} para {\code [i+1..i+m]-1}. Se aplicarmos o conceito de \textit{rolling hash}, poderemos obter este valor, nos poupando tempo no calculo da \textit{hash} e evitando que a linha $5$ seja executada em casos de chances nulas. Este é o ponto onde o algoritmo de Rabin–Karp se mostra com alto desempenho. Normalmente, o algoritmo é utilizado com uma \textit{rolling hash} simples que faz de uso  de multiplicações e somas apenas:

\begin{equation}
	H=c_1a^{k-1}+ c_2a^{k-2} + c_3a^{k-3} + ... + c_ka^{0}
\end{equation}

Onde $a$ é uma constante e $c_1,...c_k$ são os caracteres de entrada. Para evitar o calculo de números muito grandes, normalmente é usado o módulo de $m$ para o valor dos índices. A chave para uma boa \textit{hash} esta na escolha para os valores de $m$ e $a$ que satisfaçam a arquitetura utilizada para os cálculos e que se enquadrem nas condições recomendadas para uma congruência linear\cite{knuth1998art}, o que nessa \textit{hash} implica que $0<a<m$.

% subsubsection* rabin_karp (end)

\subsubsection*{Knuth–Morris–Pratt (KMP)} % (fold)
\label{ssub:knuth_morris_pratt_}

O algoritmo publicado em 1977 tem autoria de  Donald Knuth, Vaughan Pratt e James H. Morris faz de uso do conceito de autômato de estados, ou seja, uma máquina abstrata que deve estar em um, e apenas um, de seus finitos estados\cite{thierbach1985finite}. Este algoritmo segue do pressuposto que o texto é um fluxo continuo\cite{paulo2015algoritmos}, evitando de voltar a pontos anteriores do texto após uma falha de verificação, mas sim executar os seguintes passos quando encontrarmos um conflito entre {\code txt[i]} e {\code pat[j]}:

\begin{enumerate}
	\item Localizar o maior $k$ tal que {\code pat[0..k-1]} é igual a {\code txt[i-k+1..i]}.
	\item Seguir a comparação de {\code txt[i+1..]} com {\code pat[k..]}.
\end{enumerate}

Por considerar sempre um fluxo continuo do texto, a ordem de complexidade do algorítimo de Knuth–Morris–Pratt é previsível: $\Theta(n)$, porém requer um pré-processamento com ordem $\Theta(m)$ \cite{hume1991fast}.

\begin{lstlisting}[language=Bash,label=kmpalgoritm,caption={Knuth–Morris–Pratt (KMP)}]
function KMP(string S, string pattern)
	while m + i < length(S) do
	    if pattern[i] = S[m + i] then
	        if i = length(pattern) - 1 then
	            return m
	        i:= i + 1
	    else
	        if T[i] > -1 then
	            m := m + i - T[i]
	            i := T[i]
	        else
	            i := 0
	            m := m + 1
	return not found
\end{lstlisting}

No código \ref{kmpalgoritm} podemos observar o algorítimo de Knuth–Morris–Pratt genericamente aplicado.

% subsubsection* knuth_morris_pratt_ (end)
% subsection algoritimos_exatos (end)

\subsection{Algorítimos de \textit{string matching} inexatos} % (fold)

\subsubsection*{Leveinstein} % (fold)
\label{sec:leveinstein}

Duas das abordagens propostas neste trabalho são baseadas nos trabalhos de \textit{Vladimir Levenshtein}. Atuante na área de teoria da informação, código de correção de erros e analise combinatória, Levenshtein publicou 1965 na \textit{Problemy Peredachi Informatsii} o trabalho pelo qual seria reconhecido desde então: \textit{Binary codes with correction of drooping and insertion of the symbol 1}\cite{levenshtein1965}. Republicado num resumo em 1966 com o título de \textit{Binary codes capable of correcting deletions, insertions and reversals} \cite{levenshtein1966}, a obra tratava do cálculo da distância entre duas \textit{strings}, ou seja, dada as \textit{strings} \textbf{A} e \textbf{B}, determinar a quantidade mínima de operações necessárias para que, partindo da \textit{string} \textbf{A}, seja possível obter a \textit{string} \textbf{B}. Entende-se por operações a inserção, remoção e substituição de um caractere.

\paragraph*{Exemplo de cálculo de distância entre \textit{strings}} % (fold)
\label{sub:exemplo_de_c_lculo_de_distancia_entre_it}

Para uma maior compreensão do que se trata uma operação e a distância entre \textit{strings}, tomemos como exemplos duas palavras, \textbf{cantar} e \textbf{dançar}. Vamos aplicar algumas operações para, a partir da primeira, alcançar a segunda palavra.


\begin{enumerate}[start=0]
	\item cantar
	\item dantar \textit{(Substituição de ``c'' por ``d'')}
	\item dançar \textit{(Substituição de ``t'' por ``ç'')}
\end{enumerate}

Como podemos observar, foram necessárias duas operações para alcançar a palavra \textbf{dançar}. Como este é o número mínimo de operações para se obter o resultado desejado, a distância entre elas é $2$.
É importante observar observar que, mesmo que duas letras iguais e consecutivas sejam substituídas por outras duas letras idênticas, serão creditadas duas operações, por exemplo as palavras \textbf{correr} e \textbf{Potter}:

\begin{enumerate}[start=0]
	\item correr
	\item Porrer \textit{(Substituição de ``c'' por ``P'')}
	\item Potrer \textit{(Substituição do primeiro ``r'' por ``t'')}
	\item Potter \textit{(Substituição do segundo ``r'' por ``t'')}
\end{enumerate}

Precisamos então de 3 operações para alcançar \textbf{Potter} partindo de \textbf{correr}. É do pressuposto que o algoritmo é \textit{case-sensitive} e sua ordenação é baseada na Tabela \textit{ASCII (American Standard Code for Information Interchange)}, tabela inicialmente baseada no alfabeto inglês com codificação de \textit{7-bits} por caractere em um intervalo de até 128 caracteres - imprimíveis ou não \cite{shirey2007rfc}.

% TODO: Inserir uma referência para a tabela ASCII (procurar o RFC que propôs a tabela ASCII) %  DONE

\paragraph*{Distância de Levenshtein}
\label{sub:algoritmo_da_distancia}

Nos trabalhos de Levenshtein de 1965 e 1966, são apesentadas apenas as possibilidades de localização de um caractere perdido, ou do \textit{bit} perdido na sequência binária. Porém a descrição apresentada para a identificação do caractere no trabalho de 1966 \cite{levenshtein1966} possibilitou a implementação do algoritmo no decorrer da história, tendo uma das sua representações mostrada no \autoref{levenshtein_distance_py}.

% TODO: colocar o código fonte em arquivo separado e adicionar
\begin{lstlisting}[language=Python,label=levenshtein_distance_py,caption={Implementação da distância de Levenshtein}]
def calculate_distance(s1,s2):
    d = {}
    lenstr1 = len(s1)
    lenstr2 = len(s2)
    for i in xrange(-1,lenstr1+1):
        d[(i,-1)] = i+1
    for j in xrange(-1,lenstr2+1):
        d[(-1,j)] = j+1
 
    for i in xrange(lenstr1):
        for j in xrange(lenstr2):
            if s1[i] == s2[j]:
                cost = 0
            else:
                cost = 1
            d[(i,j)] = min(
                           d[(i-1,j)] + 1, # deletion
                           d[(i,j-1)] + 1, # insertion
                           d[(i-1,j-1)] + cost, # substitution
                          )

    return d[lenstr1-1,lenstr2-1]
\end{lstlisting}

Como pode-se observar, a implementação do cálculo da distância utiliza programação dinâmica a partir de operações de inserção, remoção e substituição. Tal abordagem facilita a inserção de novas operações, conforme apresentado na próxima subseção.
% subsubsection* leveinstein (end)

\subsubsection*{Damerau-Levenshtein} % (fold)
\label{sec:damerau_levenshtein}

A distância de Damerau-Levenshtein é uma variação da distância de Levenshtein \cite{levenshtein1965}, havendo ainda a adição da transposição no \textit{set} de operações básicas. Em seus estudos, Damerau apresentou que mais de $80\%$ dos erros de escrita oriundo de humanos estavam relacionados às operações de falta de caractere, excesso de caracteres, substituição de caracteres ou transposição de caracteres \cite{damerau1964technique}.


Por  considerar  a transposição de dois caracteres adjacentes, a distância de \textit{Damerau-Levenshtein} abriu portas para um novo padrão de métricas, as biológicas, no campo de mensurar a variação entre moléculas de DNA. Esta seria a brecha necessária para que surgissem outros métodos de analise de moléculas de DNA baseadas na comparação de \textit{strings}, tais como o algoritmo de \textit{Needleman–Wunsch} \cite{needleman1970general} e  o de \textit{Smith-Waterman} \cite{smith1981identification}, apresentado  na Seção \ref{sec:smith_waterman}.

\paragraph*{Implementação} % (fold)
\label{sub:implementa_damerau_levenshtein}

Para melhor contextualização da diferença entre o método tradicional de cálculo de distância e o modelo de \textit{Damerau-Levenshtein}, consideremos a seguir uma visão atômica de como seria o procedimento do cálculo da distância de Levenshtein, conforme mostrado no \autoref{levenshtein_distance_py}.

\begin{lstlisting}[language=Python,label=levenshtein_distance_py,caption={Visão atômica do cálculo da distância de Levenshtein}]
def levenshtein_distance(s1, s2):
    lenstr1 = len(s1)
    lenstr2 = len(s2)
    d = initialize_distance(lenstr1,lenstr2)

	d = calculate_distance(lenstr1,lenstr2,d) 

    return d[lenstr1-1,lenstr2-1]
\end{lstlisting}

O \autoref{levenshtein_distance_py} serviria tanto para o modelo tradicional como para o de \textit{Damerau-Levenshtein}, necessitando apenas implementar o método {\code calculate\_distance()} de forma distinta. O \autoref{damerau_levenshtein_distance_py} representa como seria a implementação para o modelo de \textit{Damerau-Levenshtein}.

\begin{lstlisting}[language=Python,label=damerau_levenshtein_distance_py,caption={Implementação da distância de Damerau-Levenshtein}]
def calculate_distance(lenstr1,lenstr2,d):
    for i in xrange(lenstr1):
        for j in xrange(lenstr2):
            if s1[i] == s2[j]:
                cost = 0
            else:
                cost = 1
            d[(i,j)] = min(
                           d[(i-1,j)] + 1, # deletion
                           d[(i,j-1)] + 1, # insertion
                           d[(i-1,j-1)] + cost, # substitution
                          )
            if i and j and s1[i]==s2[j-1] and s1[i-1] == s2[j]:
                d[(i,j)] = min (d[(i,j)], d[i-2,j-2] + cost) # transposition
 
    return d
\end{lstlisting}

A distinção do método de \textit{Damerau-Levenshtein} para o modelo tradicional é apenas a inserção das linhas \textbf{13} e \textbf{14} do \autoref{damerau_levenshtein_distance_py}, que realiza a transposição dos caracteres, quando for o caso, ao custo de uma única operação.

% subsection algoritimos_inexatos (end)