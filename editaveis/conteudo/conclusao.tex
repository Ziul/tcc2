\chapter{Conclusão} % (fold)
\label{cha:dificuldades_encontradas}


O uso de algoritmos para substituir o uso de expressões regulares apresentam ganhos de desempenho notáveis, oferecendo buscas que fazem menos uso de recursos e apresentam os resultados em menor tempo, porém a economia de memória não se mostrou  

% \section*{Guia de estilo} % (fold)
% \label{sec:guia_de_estilo}

O grupo responsável pela manutenção e evolução do {\code APT} segue um guia de estilo de código extremamente peculiar, apresentado no \autoref{styleguide}. De acordo com o guia de estilo seguido, as edentações devem ser realizadas com espaço de três para cada nível, porém a cada oito espaços, devem ser substituídos por tabulações.

% \section*{Testes quebrados} % (fold)
% \label{sec:testes_de_concorr_ncias}

Um hábito comum com o desenvolvimento da aplicação é o fato de manter \textit{builds} em que o \textit{status} na aplicação \textit{Travis CI} quebrando, devido ao fato de grande parte dos testes serem realizados com chamadas de concorrências. Devido ao tempo e ordem destas chamadas não serem deterministas, muitas vezes um teste falha por estar definido uma ordem diferente do resultado obtido pela chamada em concorrência.
Este mal habito é um dos motivos de múltiplas contribuições serem recusadas, visto que o GitHub aponta a \textit{build} como quebrada e não recomenda aceitar. Outra falha comum nos testes é um cenário para adquirir o mesmo arquivo múltiplas vezes. Este teste infelizmente falha devido a instabilidade da plataforma de integração continua (\textit{Travis CI}) que muitas das vezes resulta em um \textit{timeout} no \textit{download} do arquivo, gerando a falha no teste.


% \section*{Coesão e Coerência} % (fold)
% \label{sec:coes_o_e_coer_ncia}

Para um software de grande responsabilidade como o APT, esperava-se classes pequenas com métodos objetivos e coesos, porém é comum observar métodos e funções com mais de trinta linhas com múltiplas obrigações. Como consequência, há funções com mesmos objetivos ou semelhantes, provavelmente devido a dificuldade em se encontrar um método dedicado para uma única tarefa.

% chapter dificuldades_encontradas (end)