\chapter{Considerações Finais} % (fold)
\label{cha:dificuldades_encontradas}


O uso dos algoritmos selecionados  para substituir o uso de expressões regulares apresentaram ganhos de desempenho notáveis, oferecendo buscas que fazem menos uso de recursos e apresentam os resultados em menor tempo; Entretanto, o us de memória não mostrou  alterações significativas.

% \section*{Guia de estilo} % (fold)
% \label{sec:guia_de_estilo}

O grupo responsável pela manutenção e evolução do {\code APT} segue um guia de estilo de código extremamente peculiar, apresentado no \autoref{styleguide}. De acordo com o guia de estilo seguido, as indentações devem ser realizadas com espaço de três para cada nível, porém a cada oito espaços, os mesmos devem ser substituídos por tabulações.

% \section*{Testes quebrados} % (fold)
% \label{sec:testes_de_concorr_ncias}

Um hábito comum com o desenvolvimento do {\code APT} é o fato de manter \textit{builds} em que o \textit{status} na aplicação \textit{Travis CI} quebrando, devido ao fato de grande parte dos testes serem realizados com chamadas concorrentes. Devido ao tempo e ordem destas chamadas não serem determinísticas, muitas vezes um teste falha por estar definido uma ordem diferente do resultado obtido pela chamada paralela.
Este mal habito foi um dos motivos pelos quais múltiplas contribuições foram e vem sendo recusadas, visto que o GitHub aponta a \textit{build} como quebrada e recomenda a recusa. Outra falha comum nos testes é um cenário onde é necessário adquirir o mesmo arquivo múltiplas vezes. Este teste falha devido a instabilidade da plataforma de integração continua (\textit{Travis CI}), que muitas das vezes resulta em um \textit{timeout} no \textit{download} do arquivo, gerando a falha no teste.


% \section*{Coesão e Coerência} % (fold)
% \label{sec:coes_o_e_coer_ncia}

Para um software de grande responsabilidade como o APT, esperava-se classes pequenas com métodos objetivos e coesos, porém é comum observar métodos e funções com mais de trinta linhas com múltiplas obrigações. Como consequência, há funções com mesmos objetivos ou objetivos semelhantes, provavelmente devido a dificuldade em se encontrar um método dedicado para uma única tarefa.

Mesmo não aceitas ainda até a entrega do trabalho, as contribuições ofereciam melhorias para os usuários que desejavam poder ordenar a listagem de algum pacote não apenas de acordo com a ordem alfabética, oferecendo mais liberdade. Além de que, buscas com resultados vazios não existem mais, já que esta situação passa a ser considerada como um possível erro ortográfico do usuário e os pacotes com escrita mais próxima do escrito são apresentados.

Para o autor, trabalhar com um software livre de tamanho impacto como o {\code APT} ofereceu experiencias que não poderiam ser alcançadas facilmente com projetos pequenos. A necessidade de manter a compatibilidade com diversas arquiteturas, a taxa de cobertura de código não apenas alta, mas por vezes extensa na tentativa de simular todas as possíveis combinações de entradas para o programa. A rigorosidade de que uma funcionalidade não pode ser alterada se sua documentação não for atualizada também demonstra a preocupação dos desenvolvedores em garantir que o usuário final terá meios de conhecer melhor a ferramenta a fim de usufruir de todos os seus recursos. Ou a fidelidade ao guia de estilo de desenvolvimento, que por mais que pareça sem sentido, a equipe respeita e preserva. Porém pontos ruins foram observados como a forma que o grupo se comporta, se mantendo fechado a novas opções, seja por ser contrario a elas, ou simplesmente por ter a necessidade de manter o padrão onde a decisão deve ser tomada em grupo e só quando há empate de opiniões ou duvidas sobre determinada funcionalidade que o criador da ferramenta pode utilizar sua influencia para a tomada de decisões.
% chapter dificuldades_encontradas (end)

\section*{Trabalhos Futuros} % (fold)
\label{sec:trabalhos_futuros}

Apesar de ser um programa maduro e com alta confiabilidade pelos usuário, ainda existem diversas formas de contribuir com o {\code APT}. A contribuição mais simples esta justamente nas traduções, visto que muito das funcionalidades ainda não foram traduzidas. Porém há espaço para contribuir desenvolvendo código sim no {\code ATP}. Em diversos pontos do programa pode-se ver um acoplamento muito grande de métodos e funções, com arquivos com mais de 300 linhas contendo apenas uma função, ou a duplicação de rotinas devido a má estruturação dos arquivos e funções. Na própria busca por pacotes podemos observar uma duplicidade de afazeres quando a requisição à \textit{cache} de pacotes gera uma ordenação antes do envio dos pacotes, para uma função que irá reordenar os pacotes antes de imprimir a mensagem.

% section trabalhos_futuros (end)